% ===============================================================================================
% LaTeX Presentation Template by Lorenz Vogel (version 2022-03), adapted by Bjoern Malte Schaefer
% ===============================================================================================

\documentclass[aspectratio=169, xcolor=dvipsnames, 12pt, t]{beamer}

% packages
\usepackage{adjustbox}
\usepackage{ifthen}
\usepackage{pgfplots}
\usepackage{algorithm} 
\usepackage{multimedia}
\usepackage{media9}

\usepgfplotslibrary{patchplots}
\usetikzlibrary{patterns, positioning, arrows}

% inputs
% ========================================================================================
% LaTeX Presentation Template by Lorenz Vogel (version 2022-03)
% Plehn Group, Institute for Theoretical Physics, Heidelberg University
% [compiler: pdfLaTeX]
% ========================================================================================

\usepackage[utf8]{inputenc} % input encodings (allow UTF-8 input)
\usepackage[T1]{fontenc} 	% selecting font encodings (use 8-bit T1 fonts)
\usepackage[english]{babel} % multilingual support (English language/hyphenation)
%\usepackage{lmodern} 		% Latin Modern fonts
\usepackage{geometry} 		% flexible and complete interface to document dimensions
%\usepackage{microtype} 		% subliminal refinements towards typographical perfection
\usepackage{amsmath} 		% math package (American Mathematical Society)
\usepackage{amssymb} 		% math package (extended symbol collection)

\newcommand\hmmax{0}
\newcommand\bmmax{0}

\usepackage{amsthm} 		% math package (typesetting theorems using AMS style)
\usepackage{mathtools} 		% math package (fixes various deficiencies of amsmath)
\usepackage{float} 			% floating objects such as figures and tables
\usepackage{graphicx} 		% enhanced support for graphics
\usepackage{tabularx} 		% tabulars with adjustable-width columns
\usepackage{booktabs} 		% professional-quality tables
\usepackage{pdfpages} 		% inclusion of external multi-page PDF documents
\usepackage{extarrows} 		% extra arrows beyond those provided in amsmath
\usepackage{multirow} 		% create tabular cells spanning multiple rows
\usepackage{multicol} 		% intermix single and multiple columns
\usepackage{caption} 		% customising captions in floating environments
\usepackage{subcaption} 	% support for sub-captions (captions for subfigures)
\usepackage{setspace} 		% set space between lines (single, onehalf, double)
\usepackage{xspace} 		% define commands that appear not to eat spaces
\usepackage{ragged2e} 		% alternative versions of ragged-type commands
\usepackage{stackrel} 		% enhancement to the \stackrel command
\usepackage{tikz} 			% create PostScript and PDF graphics
\usepackage{braket} 		% Dirac bra-ket notation
\usepackage{bm} 			% access bold symbols in maths mode
\usepackage{tensor} 		% typeset tensors (tensor-style super- and subscripts)
\usepackage{slashed} 		% slash through characters (Feynman slash notation)
\usepackage{siunitx} 		% SI units package (typesetting values with units)
%\usepackage{lastpage} 		% reference last page
\usepackage{cite} 			% improved citation handling
\usepackage[normalem]{ulem} % package for underlining
%\usepackage{fontawesome} 	% access to web-related icons
\usepackage{doi} 			% create correct hyperlinks for DOI numbers
\usepackage{hyperref} 		% hypertext links (handel cross-referencing commands)
\usepackage[dvipsnames]{xcolor} 				% colour management
\usepackage[most]{tcolorbox} 					% coloured and framed text boxes
%\usepackage[nameinlink, capitalize]{cleveref} 	% intelligent cross-referencing
%\usepackage[ruled, vlined]{algorithm2e} 		% floating algorithm environment
\usepackage{pgfpages}
\usepackage{isotope}
\usepackage{pifont}
\usepackage{appendixnumberbeamer}
\usepackage[version=4]{mhchem}
\usepackage[compat=1.0.0]{tikz-feynman}



%\usetheme{metropolis}
%\usecolortheme{dove}
%\usepackage[default, lining]{FiraSans}
%\usefonttheme[onlymath]{serif}

\usetheme{metropolis}
\usecolortheme{dove}
\usefonttheme[bitstream-charter]{serif}
\usepackage[bitstream-charter]{mathdesign}
\usefonttheme{professionalfonts}


\metroset{
	titleformat=regular, % regular, smallcaps, allsmallcaps, allcaps
	sectionpage=progressbar, % none, simple, progressbar
	subsectionpage=none, % none, simple, progressbar
	numbering=fraction, % none, counter, fraction
	progressbar=frametitle, % none, head, frametitle, foot
	%background=light, % dark, light
}




\newcommand{\cmark}{\ding{51}}
\newcommand{\xmark}{\ding{55}}

% prevent all line breaks in inline equations
\binoppenalty=10000
\relpenalty=10000

% fix \cal and \mathcal characters look (so it's not the same as \mathscr)
\DeclareSymbolFont{usualmathcal}{OMS}{cmsy}{m}{n}
\DeclareSymbolFontAlphabet{\mathcal}{usualmathcal}




\hypersetup{
	pdftitle={Bachelor Thesis Presentation},
	pdfauthor={David Li},
	colorlinks=true, 	% false: boxed links, true: colored links
	linkcolor={black}, 	% color of internal links (sections, pages, etc.)
	citecolor={black}, 	% color of citation links (links to bibliography)
	urlcolor={black}, 	% color of URL links (external links)
} 
% more hypersetup options: 
% linktoc=none,section,page,all (defines which part in the TOC is made into a hyperlink)
% hidelinks (removing color and border)





% algorithm environment settings
%\renewcommand{\algorithmautorefname}{Algorithm}
%\newcommand\mycommfont[1]{\textcolor{gray}{#1}}
%\SetArgSty{textnormal}
%\SetKwComment{Comment}{{\small\#}~}{}
%\SetCommentSty{mycommfont}

%\renewcommand{\thefootnote}{\fnsymbol{footnote}} 	% footnote symbol
%\setlength{\tabcolsep}{5pt}		% adding space between columns in a table
%\setlength{\parskip}{5pt} 		% parameter that characterises the paragraph spacing
%\setlength{\parindent}{0pt} 	% parameter that characterises the paragraph indentation

\usetikzlibrary{arrows, arrows.meta, shapes, calc} 	% load customized packages and settings
% ========================================================================================
% LaTeX Presentation Template by Lorenz Vogel (version 2022-03)
% Plehn Group, Institute for Theoretical Physics, Heidelberg University
% [compiler: pdfLaTeX]
% ========================================================================================

% corporate design colors: signet background (lightred) and signet outline (darkred)
% [https://www.uni-heidelberg.de/en/university/about-the-university/corporate-design]
\definecolor{unihd-lightred}{cmyk}{0.20, 1.00, 0.90, 0.00} % C20 / M100 / Y90 / K0
\definecolor{unihd-darkred}{cmyk}{0.30, 1.00, 0.90, 0.65}  % C30 / M100 / Y90 / K65
\definecolor{unihd-sand}{cmyk}{0.05, 0.05, 0.09, 0.00}     % C5  / M5   / Y9  / K0
\colorlet{unihd-bkgred25}{rgb:unihd-lightred!25,1}
\colorlet{unihd-bkgred40}{rgb:unihd-lightred!40,1}

% harmonic colors
\definecolor{TorchRed}{cmyk}{0.00, 1.00, 0.80, 0.05}
\definecolor{Carmine}{cmyk}{0.00, 1.00, 0.81, 0.33}
\definecolor{TickleMePink}{cmyk}{0.00, 0.50, 0.40, 0.00}
\definecolor{Green}{cmyk}{0.89, 0.00, 1.00, 0.11}
\definecolor{JapaneseLaurel}{cmyk}{0.89, 0.00, 1.00, 0.38}
\definecolor{SnowyMint}{cmyk}{0.22, 0.00, 0.25, 0.00}
\definecolor{MintGreen}{cmyk}{0.45, 0.00, 0.50, 0.00}
\definecolor{DeepCerulean}{cmyk}{1.00, 0.24, 0.00, 0.35}
\definecolor{Orient}{cmyk}{1.00, 0.24, 0.00, 0.55}
\definecolor{Onahau}{cmyk}{0.25, 0.06, 0.00, 0.00}
\definecolor{Anakiwa}{cmyk}{0.50, 0.12, 0.00, 0.00}
\definecolor{FlushOrange}{cmyk}{0.00, 0.55, 1.00, 0.00}
\definecolor{RoseOfSharon}{cmyk}{0.00, 0.55, 1.00, 0.30}
\definecolor{Negroni}{cmyk}{0.00, 0.14, 0.25, 0.00}
\definecolor{MacaroniAndCheese}{cmyk}{0.00, 0.27, 0.50, 0.00}

% red, green, and blue comments
\newcommand{\red}[1]{\textcolor{unihd-lightred}{#1}}
\newcommand{\blue}[1]{\textcolor{DeepCerulean}{#1}}
\newcommand{\green}[1]{\textcolor{Green}{#1}}




\newtcolorbox{cblock1}[2][]{%
	coltitle=black,
	colback=unihd-bkgred25,
	colframe=unihd-lightred,
	colbacktitle=unihd-bkgred25,
	boxrule=1pt,
	titlerule=1pt,
	arc=4pt,
	left=2pt,
	right=2pt,
	top=2pt,
	bottom=2pt,
	fonttitle=\bfseries,
	title={#2},
	%sharp corners=downhill, %northwest
	#1}




\setbeamertemplate{itemize item}[circle]
\setbeamertemplate{itemize subitem}[circle]
\setbeamertemplate{enumerate item}{\insertenumlabel.}
\setbeamertemplate{enumerate subitem}{\insertenumlabel.\insertsubenumlabel}
\setbeamertemplate{frame footer}{%
	\insertshortauthor~---~\insertshortinstitute~\vspace{-6pt}}
\addtobeamertemplate{footline}{\hypersetup{allcolors=.}}{}


\AtBeginNote{%
	\setbeamertemplate{itemize item}[square]
	\setbeamertemplate{itemize subitem}[circle]
	\setbeamercolor{itemize item}{fg=black}
	\setbeamercolor{itemize subitem}{fg=black}
	\setbeamercolor{enumerate item}{fg=black}
	\setbeamercolor{enumerate subitem}{fg=black}
	\setlength{\leftmargini}{0.5cm}
	\setlength{\leftmarginii}{0.5cm}
}



\setbeamercolor{title separator}{fg=unihd-lightred}
\setbeamercolor{progress bar}{fg=unihd-lightred, bg=unihd-bkgred25}
\setbeamercolor{frame footer}{fg=gray}
\setbeamercolor{frame numbering}{fg=gray}
%\setbeamercolor{progress bar in head/foot}{fg=unihd-lightred, bg=unihd-bkgred25}
%\setbeamercolor{progress bar in section page}{fg=unihd-lightred, bg=unihd-bkgred25}

\setbeamercolor{itemize item}{fg=unihd-lightred}
\setbeamercolor{itemize subitem}{fg=unihd-lightred}
\setbeamercolor{enumerate item}{fg=unihd-lightred}
\setbeamercolor{enumerate subitem}{fg=unihd-lightred}

\setbeamercolor{title}{fg=black}
\setbeamercolor{subtitle}{fg=black}
\setbeamercolor{author}{fg=black}
%\setbeamercolor{date}{fg=black}
%\setbeamercolor{institute}{fg=black}


% for [10pt]:
% \tiny 5
% \scriptsize 7
% \footnotesize 8
% \small 9
% \normalsize 10
% \large 12
% \Large 14.4
% \LARGE 17.28
% \huge 20.74
% \Huge 24.88

\setbeamerfont{title}{size=\fontsize{15.0}{18.0}, series=\bfseries}
\setbeamerfont{subtitle}{size=\large}
\setbeamerfont{author}{size=\fontsize{11.0}{13.2}, series=\bfseries}
\setbeamerfont{date}{size=\small}
\setbeamerfont{institute}{size=\small}

\setbeamerfont{frametitle}{size=\fontsize{12.5}{15.0}, series=\bfseries}
\setbeamerfont{frame footer}{size=\tiny} % \scriptsize
\setbeamerfont{frame numbering}{size=\tiny} % \footnotesize
\setbeamerfont{block title}{size=\normalsize, series=\bfseries}
\setbeamerfont{block body}{size=\normalsize}
\setbeamerfont{caption}{size=\small}
\setbeamerfont{footnote}{size=\small}

%\setbeamerfont{itemize/enumerate body}{size=\normalsize}
%\setbeamerfont{itemize/enumerate subbody}{size=\normalsize}
%\setbeamerfont{itemize/enumerate subsubbody}{size=\normalsize}




\makeatletter
\setlength{\metropolis@progressinheadfoot@linewidth}{1.5pt}
\setlength{\metropolis@titleseparator@linewidth}{2pt}
\setlength{\metropolis@progressonsectionpage@linewidth}{2pt}

\setbeamertemplate{frametitle}{
	\nointerlineskip\vspace{5pt}
	\begin{beamercolorbox}[
		wd=\paperwidth, sep=0pt,
		leftskip=\metropolis@frametitle@padding,
		rightskip=\metropolis@frametitle@padding,
		]{frametitle}
			\metropolis@frametitlestrut@start
			\insertframetitle
			%\nolinebreak
			\metropolis@frametitlestrut@end
	\end{beamercolorbox}
	\par\vspace{-8pt}
	\usebeamertemplate*{progress bar in head/foot}
}

\setbeamertemplate{progress bar in head/foot}{
	\nointerlineskip
	\setlength{\metropolis@progressinheadfoot}{
		0.86\paperwidth*\ratio{\insertframenumber pt}{\inserttotalframenumber pt}}
	\hspace{-22.5pt}
	\begin{beamercolorbox}[wd=0.86\paperwidth]{progress bar in head/foot}
		\begin{tikzpicture}
			\fill[bg] (0,0) rectangle (0.86\paperwidth, 
			\metropolis@progressinheadfoot@linewidth);
			\fill[fg] (0,0) rectangle (\metropolis@progressinheadfoot, 
			\metropolis@progressinheadfoot@linewidth);
		\end{tikzpicture}
	\end{beamercolorbox}
}

\makeatother




%\AtBeginSection[]{%
%	\setbeamertemplate{footline}{\hbox{}}
%	\begin{frame}<beamer>[noframenumbering]
%		\frametitle{Outline}
%		\setbeamertemplate{section in toc}[sections numbered]
%		\tableofcontents[currentsection]
%	\end{frame}
%} 		% load customized beamer settings
\newcommand{\ie}{i.e.\@\xspace} 		% id est (i.e.)
\newcommand{\eg}{e.g.\@\xspace} 		% exempli gratia (e.g.)
\newcommand{\vs}{vs.\@\xspace} 			% versus (vs.)
\newcommand{\etal}{et~al.\@\xspace}		% et al.
\newcommand{\eqcomma}{\quad\text{,}} 	% equation comma
\newcommand{\eqperiod}{\quad\text{.}} 	% equation period

% red, green, and blue comments
%\newcommand{\red}[1]{\textcolor{unihd-lightred}{#1}}
%\newcommand{\blue}[1]{\textcolor{blue}{#1}}
%\newcommand{\green}[1]{\textcolor{green}{#1}}

% brackets and parentheses
\newcommand{\Langle}{\bigl\langle}
\newcommand{\Rangle}{\bigr\rangle}
\newcommand{\XLangle}{\Bigl\langle}
\newcommand{\XRangle}{\Bigr\rangle}
\newcommand{\XXLangle}{\biggl\langle}
\newcommand{\XXRangle}{\biggr\rangle}
\newcommand{\Lbrack}{\bigl\lbrack}
\newcommand{\Rbrack}{\bigr\rbrack}
\newcommand{\XLbrack}{\Bigl\lbrack}
\newcommand{\XRbrack}{\Bigr\rbrack}
\newcommand{\XXLbrack}{\biggl\lbrack}
\newcommand{\XXRbrack}{\biggr\rbrack}
\newcommand{\Lbrace}{\bigl\lbrace}
\newcommand{\Rbrace}{\bigr\rbrace}
\newcommand{\XLbrace}{\Bigl\lbrace}
\newcommand{\XRbrace}{\Bigr\rbrace}
\newcommand{\XXLbrace}{\biggl\lbrace}
\newcommand{\XXRbrace}{\biggr\rbrace}
\newcommand{\braces}[1]{\lbrace#1\rbrace}

\newcommand{\qqquad}{\qquad\quad}
\newcommand{\qqqquad}{\qquad\qquad}

% general math definitions
\newcommand{\diff}{\mathop{}\!\mathrm{d}} 	% differential
\newcommand{\del}{\partial} 				% partial derivative
\newcommand{\vect}[1]{\bm{#1}} 				% bold vector notation
\newcommand{\abs}[1]{\lvert#1\rvert} 		% absolute value (single vertical lines)
\newcommand{\norm}[1]{\lVert#1\rVert} 		% norm (double vertical lines)
\newcommand{\Z}{\mathbb{Z}} 				% integers
\newcommand{\Q}{\mathbb{Q}} 				% rational numbers
\newcommand{\R}{\mathbb{R}} 				% real numbers
\newcommand{\var}{\operatorname{Var}} 		% variance
\newcommand{\sign}{\operatorname{sign}} 	% sign
\newcommand{\tr}{\operatorname{Tr}}			% trace
\newcommand{\order}{\mathcal{O}} 			% order
\newcommand{\imag}{\mathrm{i}} 				% imaginary unit
\newcommand{\euler}{\mathrm{e}} 			% Euler's number
\newcommand{\iid}{\text{i.i.d.}} 			% independent and identically distributed
\newcommand{\range}[2]{#1,\ldots,#2}
\newcommand{\really}{\stackrel{!}{=}}
\newcommand{\mean}[1]{\left\langle#1\right\rangle}

% particle physics and machine learning definitions
\newcommand{\similarity}{\operatorname{sim}}
\newcommand{\softmax}{\operatorname{softmax}}
\newcommand{\sigmoid}{\operatorname{sigmoid}}
\newcommand{\ReLU}{\operatorname{ReLU}}
\newcommand{\LeakyReLU}{\operatorname{LeakyReLU}}
\newcommand{\Attention}{\operatorname{Attention}}
\newcommand{\MultiHead}{\operatorname{MultiHead}}
\newcommand{\Concat}{\operatorname{Concat}}
\newcommand{\head}{\operatorname{head}}

\newcommand{\pT}{p_{\mathrm{T}}} 	% transverse momentum
\newcommand{\pTi}{p_{\mathrm{T},i}} % transverse momentum (with index i)
\newcommand{\pTjet}{p_{\mathrm{T},\,\mathrm{jet}}}
\newcommand{\etajet}{\eta_{\,\mathrm{jet}}}
\newcommand{\yjet}{y_{\,\mathrm{jet}}}
\newcommand{\kT}{k_{\mathrm{T}}}
\newcommand{\mT}{m_{\mathrm{T}}} 	% transverse mass
\newcommand{\sqrts}{\sqrt{s}} 		% center-of-mass energy
\newcommand{\loss}{\mathcal{L}} 	% loss value
\newcommand{\dth}{\Delta_{\mathrm{th/sys}}}
\newcommand{\dst}{\Delta_{\mathrm{stat}}}
\newcommand{\dsy}{\Delta_{\mathrm{syst}}}

\newcommand{\pTa}{p_{\mathrm{T},a}} % transverse momentum (with index a)
\newcommand{\pTb}{p_{\mathrm{T},b}} % transverse momentum (with index b)
\newcommand{\batch}{\mathrm{batch}}
\newcommand{\Nbatch}{N_{\mathrm{batch}}}
\newcommand{\nc}{n_{C}} % number of constituents

\newcommand{\etaphi}{$\eta$--$\phi$\xspace} % eta-phi plane
\newcommand{\AUC}{\mathrm{AUC}} % area under the ROC curve
\newcommand{\es}{\epsilon_{s}} 	% signal efficiency (true-positive rate)
\newcommand{\eb}{\epsilon_{b}} 	% background mistag rate (false-positive rate)
\newcommand{\imtafe}{\epsilon_{b}^{-1}(\epsilon_{s}\!=\!0.5)}
\newcommand{\sklearn}{\texttt{scikit-learn}\xspace}
\newcommand{\PyTorch}{\texttt{PyTorch}\xspace}
\newcommand{\Adam}{\texttt{Adam}\xspace}
\newcommand{\AdamW}{\texttt{AdamW}\xspace}
\newcommand{\EnergyFlow}{\texttt{EnergyFlow}\xspace}
\newcommand{\FastJet}{\textsc{FastJet}\xspace}
\newcommand{\Pythia}{\textsc{Pythia}\xspace}
\newcommand{\Delphes}{\textsc{Delphes}\xspace}
\newcommand{\Rivet}{\textsc{Rivet}\xspace}
\newcommand{\Python}{\texttt{Python}\xspace}
\newcommand{\NumPy}{\texttt{NumPy}\xspace}
\newcommand{\Vegas}{\textsc{Vegas}\xspace}
\newcommand{\Madgraph}{\textsc{Madgraph}\xspace}
\newcommand{\Sherpa}{\textsc{Sherpa}\xspace}
\newcommand{\Keras}{\textsc{Keras}\xspace}

\newcommand{\EFP}{\mathrm{EFP}} 		% energy flow polynomials
\newcommand{\ttbar}{t\bar{t}} 			% top-antitop pair
\newcommand{\electron}{\mathrm{e}^{-}} 	% electron
\newcommand{\positron}{\mathrm{e}^{+}} 	% positron

% hyperlink references
\newcommand{\urlx}[1]{\href{#1}{#1}}
\newcommand{\DOI}[2]{\href{http://dx.doi.org/#2}{#1}}
%\newcommand{\arXiv}[1]{\href{http://arxiv.org/abs/#1}{arXiv:#1}}
\newcommand{\arXiv}[2][]{%
	\ifthenelse{\equal{#1}{}}%
	{\href{http://arxiv.org/abs/#2}{arXiv:#2}}%
	{\href{http://arxiv.org/abs/#2}{arXiv:#2~[#1]}}}
\newcommand{\arXivred}[2][]{%
	\ifthenelse{\equal{#1}{}}%
	{\href{http://arxiv.org/abs/#2}{\red{arXiv:#2}}}%
	{\href{http://arxiv.org/abs/#2}{\red{arXiv:#2~[#1]}}}} 				% load pre-defined shortcuts
\graphicspath{{./figures/}}

%\setbeameroption{show notes on second screen=right}
\setstretch{0.96}

% definitions
\newcommand{\dang}{d_\mathrm{A}}
\newcommand{\dd}{\mathrm{d}}
\newcommand{\sun}{\odot}
\newcommand{\vecj}{\vec{\jmath}}
\newcommand{\trace}{\mathrm{tr}}
\newcommand{\likelihood}{\mathcal{L}}
\newcommand{\ci}{\mathrm{i}}
\renewcommand{\tensor}[1]{\hat{#1}}
\DeclareMathOperator{\Div}{div}
\newcommand{\D}{\mathcal{D}}




\newcommand{\const}{\text{const.}}
\renewcommand{\d}{\text{d}}
\newcommand{\ddt}{\dfrac{\d}{\d t}}
\newcommand*{\defeq}{\mathrel{\vcenter{\baselineskip0.5ex \lineskiplimit0pt \hbox{\scriptsize.}\hbox{\scriptsize.}}}=} % Macht ein schönes :=
%Zahlenkörper
\newcommand{\C}{\mathbb{C}}
\newcommand{\N}{\mathbb{N}}
\newcommand{\F}{\mathcal{F}}
\renewcommand{\H}{\mathcal{H}}
\renewcommand{\L}{\mathcal{L}}
\renewcommand{\D}{\mathcal{D}}
\newcommand{\J}{\mathcal{J}}
\newcommand{\X}{\mathcal{X}}
\newcommand{\K}{\mathbb{K}}
\newcommand{\B}{\mathcal{B}}
\newcommand{\G}{\mathcal{G}}
\newcommand{\I}{\mathcal{I}}
\newcommand{\e}{\textrm{e}}
\newcommand{\seminorm}[1]{\left\lvert\hspace{-1 pt}\left\lvert\hspace{-1 pt}\left\lvert {#1}

\right\lvert\hspace{-1 pt}\right\lvert\hspace{-1 pt}\right\lvert}
% title
\title[]{Controlling electric and nematic fields through shape optimization}
\author{David Li}
\date{29th of September 2025}

\institute[Heidelberg University]{
	Heidelberg University, Germany\\
    Department of Physics and Astronomy, Institute for Theoretical Physics
    \\ \\
	\textit{Bachelor Thesis supervised by Prof. Ulrich Schwarz and Santiago Gomez Melo}\\\\\\
}

\theoremstyle{plain}
\newtheorem{proposition}{Proposition}

\begin{document}

%% title page
\begin{frame}[noframenumbering, plain]
	\begin{spacing}{0.95}
		\titlepage
	\end{spacing}
\end{frame}



\begin{frame}
	\frametitle{Shape Optimization}
	\begin{itemize}
		\item Achieve optimal physical properties  by tailoring the geometry, e.g.:
		\begin{itemize}
			\item Minimize drag in fluid dynamics
			\item Maximize structural strength
		\end{itemize}
		\item Enables scalable inverse design of shapes 
	\end{itemize}
	\begin{figure}
		\centering
		\includegraphics[width=0.5\textwidth]{AcousticHorn.png}
		\caption{Example of shape optimization of an acoustic horn maximizing transmission efficiency, left initial shape, right optimized shape (from \textit{Schmidt} et al.\textit{ SIAM Journal on Scientific Computing}, 2016)}
	\end{figure}
\end{frame}

\begin{frame}
	\frametitle{Metamaterials}
	\begin{itemize}
		\item Metamaterials: engineered materials with structure-driven properties.
		\item Metamaterials can be engineered to have tunable properties via external stimuli.
	\end{itemize}
	\begin{figure}
		\centering
		\includegraphics[width=0.4\textwidth]{Optomechanical.png}
		\caption{Examples of metamaterials with tunable properties (from \textit{Münchinger} et al. \textit{Materials Today} 2022)}
	\end{figure}
\end{frame}

\begin{frame}
	\frametitle{Liquid Crystals and Nematic Phase}
	\begin{itemize}
		\item Liquid crystals: fluid-like materials with orientational order.
		\item Liquid crystal elastomers: liquid crystals + polymer elasticity, enable large, reversible deformations.
		% \item Q-order tensor: describes how the liquid crystal molecules deviate from isotropy
		% \begin{equation*}
		% 	Q = S\left(n \otimes n - \frac{1}{3}\mathrm{Id}\right)
		% \end{equation*}
	\end{itemize}
	\begin{figure}
		\centering
		\includegraphics[width=0.8\textwidth]{LcPhases.jpg}
		\caption{Different phases of liquid crystals. (from \textit{Zang} et al. \textit{Science Advances} 2025)}
	\end{figure}
\end{frame}

\begin{frame}
	\begin{figure}
		\centering
		\includegraphics[width=\textwidth]{PictureZoom.png}
	\end{figure}
\end{frame}

\begin{frame}
	\frametitle{Liquid Crystal Elastomers (LCEs)}
		\begin{figure}[h]
				\centering
				\includegraphics[width=\textwidth]{Thermalresponse.jpg}
			\caption{Left: Schematic of LCEs in the nematic phase. Right: Thermal response of an LCE/LCN showing deformation across the nematic-isotropic transition. (From \textit{Zang} et al. \textit{Advanced Materials} 2025)}
		\end{figure}
\end{frame}
% \begin{frame}
% 	\begin{center}
% 	\begin{tikzpicture}[every node/.style={font=\large}]
% 		% Draw circles
% 		\node[draw, circle, minimum size=4cm, inner sep=0pt] (A) at (-7,2) {};
% 		\node[draw, circle, minimum size=4cm, inner sep=0pt] (B) at (-2,0) {};
% 		\node[draw, circle, minimum size=4cm, inner sep=0pt] (C) at (3,-2) {};

% 		% Connect circles
% 		\draw[very thick, -{Latex}] (A) -- (B);
% 		\draw[very thick, -{Latex}] (B) -- (C);

% 		% Place images inside circles
% 		\node[anchor=center] at (A) {\includegraphics[width=3cm]{Optomechanical.png}};
% 		\node[anchor=center] at (B) {\includegraphics[width=3cm]{OptomechanicalCloseUp.png}};
% 		\node[anchor=center] at (C) {\includegraphics[width=3cm]{LCEs.png}};

% 		% Optional: labels
% 		% \node at ($(A)+(0,1.5)$) {Metamaterial};
% 		% \node at ($(B)+(1.2,0)$) {Microstructure};
% 		% \node at ($(C)+(1.2,-0.5)$) {LCE detail};
% 	\end{tikzpicture}
% 	\end{center}
% \end{frame}
% -------------------------------------------------------- %



% \begin{frame}
% 	\frametitle{Nematic Liquid Crystals}
% 	Landau-de Gennes free energy
% 	\begin{align}
%     E_{\textrm{LdG}}(\tensor{Q}) = \frac{1}{2}\int_{\Omega} W_{\textrm{el}}(\nabla \tensor{Q})\, \d x + \frac{1}{\eta_\mathrm{B} }\int_{\Omega} \psi_{\mathrm{bulk}}(\tensor{Q})\, \d x,
% \end{align}
% where the elastic energy density is given by
% \begin{align}
%     W_{\textrm{el}}(\nabla \tensor{Q}) = L|\nabla \tensor{Q}|^2
% \end{align}
% and the bulk energy density
% \begin{align}\label{eq:bulk-free-energy}
%     \psi_{\mathrm{bulk}}(\tensor{Q}) = U_0 + \frac{A}{2} \tr(\tensor{Q}^2) + \frac{B}{3} \tr(\tensor{Q}^3) + \frac{C}{4} (\tr(\tensor{Q}^2))^2.
% \end{align}
% \end{frame}


% -------------------------------------------------------- %
\begin{frame}
	\frametitle{Alignment of Liquid Crystals}
	\begin{itemize}
		\item LC alignment: surface anchoring or external fields
		\item 3D-printed PDMS microscaffolds enable complex alignment, \textit{Hsu} et al.\cite{MeloAlignment2024}
		\item LC molecules align perpendicular to scaffold surface
	\end{itemize}
	\(\rightarrow\) Goal: optimize scaffold shape for desired LC alignment

	\begin{figure}
		\centering
		\includegraphics[width=0.5\textwidth]{MeloAlignment.png}
		\caption{3D-printed PDMS microscaffold for LC alignment (from \textit{Hsu} et al. \textit{Advanced Materials}, 2024)}
	\end{figure}
\end{frame}

% -------------------------------------------------------- %
\begin{frame}
	\frametitle{Shape Optimization}
	\begin{itemize}
		\item Shape optimization seeks the domain shape \(\Omega\) that optimizes an objective \(J(\Omega)\) under constraints (often PDEs).
		\item The shape derivative quantifies how \(J\) changes under a domain deformation \(\theta\):
		\[
		dJ(\Omega)[\theta] = \lim_{t \to 0} \frac{J(\Omega + t\theta(\Omega)) - J(\Omega)}{t}.
		\]
		% \item To infer descent directions for the optimization, we use shape gradients
	\end{itemize}
	\begin{figure}
		\centering
		\includegraphics[width=0.35\textwidth]{deformation.png}
		\caption{Illustration of a domain deformation $\Omega \mapsto \Omega + t\theta(\Omega)$ (from \textit{Allaire} et al. \textit{Handbook of Numerical Analysis}, 2021)}
	\end{figure}
\end{frame}



\begin{frame}
	\frametitle{Algorithm}
	\subsection{Algorithmic Overview}
\begin{center}
\begin{tikzpicture}[node distance=2.5cm, every node/.style={font=\small}, >=stealth]
	% Nodes
		\node[draw, rectangle, rounded corners, fill=blue!10, minimum width=3.5cm, minimum height=1cm, align=center] (init) {Initialize shape \\ $\Omega^0$};
		\node[draw, rectangle, rounded corners, fill=green!10, right=1cm of init, minimum width=3.5cm, minimum height=1cm, align=center] (state) {Solve state equation \\ for $u_{\Omega^n}$ via Finite Elements};
		\node[draw, rectangle, rounded corners, fill=yellow!10, right=1cm of state, minimum width=3.5cm, minimum height=1cm, align=center] (obj) {Evaluate objective \\ $J(\Omega^n)$};
		\node[draw, rectangle, rounded corners, fill=orange!10, below=0.7cm of obj, minimum width=3.5cm, minimum height=1cm, align=center] (grad) {Compute shape derivative \\ and gradient $\theta^n$ with an \\ elasticity inner product};
		\node[draw, rectangle, rounded corners, fill=red!10, below=0.7cm of grad, minimum width=3.5cm, minimum height=1cm, align=center] (step) {Line search \\ (Armijo) for $\alpha^n$};
		\node[draw, rectangle, rounded corners, fill=purple!10, left=1cm of step, minimum width=3.5cm, minimum height=1cm, align=center] (update) {Update shape: \\ $\Omega^{n+1} = (\mathrm{Id} + \alpha^n \theta^n)(\Omega^n)$};
		\node[draw, rectangle, rounded corners, fill=gray!20, below=0.7cm of update, minimum width=3.5cm, minimum height=1cm, align=center] (finish) {Finished};

		% Arrows
		\draw[->, thick] (init) -- (state);
		\draw[->, thick] (state) -- (obj);
		\draw[->, thick] (obj) -- (grad);
		\draw[->, thick] (grad) -- (step);
		\draw[->, thick] (step) -- (update);

		% Arrow from update to state (loop) with label
		\draw[->, thick] (update.north) -- ++(0,0.8) -| node[left, pos=0.75]{no convergence} (state.south);

		% Arrow from update to finish (converged)
		\draw[->, thick] (update) -- node[right]{converged} (finish);
\end{tikzpicture}
\end{center}
\end{frame}


\begin{frame}
	\frametitle{Algorithm: Implementation}
	\begin{itemize}
		\item Implemented in Python with (legacy) FEniCS and dolfin-adjoint in the Unified Form Language (UFL)
		\item Shape derivative: automatic shape differentiation
	\end{itemize}
\end{frame}

% \begin{frame}
% 	\frametitle{Results: Electric Field}
% 		We want to optimize the shape of a charged subdomain \(\Omega \subset \D\) to achieve a desired electric potential \(u^*\) inside the computational domain \(\D\)
% 	\begin{figure}
% 		\centering
% 		\includegraphics[width=0.5\textwidth]{EFieldStartingCharge.png}
% 		\caption{Initial computational domain \(\D\) with charged subdomain \(\Omega\) (in red).}
% 	\end{figure}
% \end{frame}

\begin{frame}
	\frametitle{Results: Electric Field}
		The objective functional is given by
		\begin{align}
			J(\Omega) = \int_\D (u(x) - u^*(x))^2 \, \d x,
		\end{align}
		where \(u^*\) is the target electric potential (of an ellipse) and \(u\) is subject to the Poisson equation
		\begin{align}
			-\Delta u(x) = f(x) \quad \text{in } \mathcal{D}, \\
			u(x) = 0 \quad \text{on } \partial \mathcal{D},
		\end{align}
		with the source term and charge \(Q\)
		\begin{align}
			f(x) = \begin{cases}
				Q, & x \in \Omega, \\
				0, & x \notin \Omega.
			\end{cases}
		\end{align}
\end{frame}

\begin{frame}
	\frametitle{Results: Electric Field}
	We calculated the target potential \(u^*\) by solving the Poisson equation with a elliptical charged domain \(\Omega^*\) 
	\begin{figure}
		\centering
		\includegraphics[width=0.5\textwidth]{EFieldTargetCharge.png}
		\caption{Computational domain \(\D\) with charged (target) subdomain \(\Omega^*\) (in red).}
	\end{figure}
\end{frame}


% \begin{frame}
% 	\frametitle{Results: Electric Field}
% 	\begin{figure}[ht]
% 		\centering
% 		\begin{minipage}{0.48\textwidth}
% 			\centering
% 			\includegraphics[width=\textwidth]{Example3It6.png}
% 			\caption*{Final shape using $H^1$ shape gradient with mesh penalty}
% 		\end{minipage}\hfill
% 		\begin{minipage}{0.48\textwidth}
% 			\centering
% 			\includegraphics[width=\textwidth]{Example4It25.png}
% 			\caption*{Final shape using elasticity shape gradient}
% 		\end{minipage}
% 		\caption{Comparison of optimized shapes after the final iteration for both approaches.}
% 	\end{figure}
% \end{frame}


\begin{frame}
	\frametitle{Results: Electric Field}
	\begin{figure}
		\centering
		\includegraphics[width=0.5\textwidth]{Example4Meshquality.png}
		\caption{Final shape using elasticity inner product with mesh quality issues displayed}
	\end{figure}
\end{frame}

% \begin{frame}
% 	\frametitle{Results: Electric Field}
% 	We see mesh quality issues that are two-fold:
% 	\begin{enumerate}
% 		\item Intersecting mesh cells
% 		\item Distorted mesh cells
% 	\end{enumerate}
% 	Stategies to counter that include
% 	\begin{itemize}
% 		\item Smoothing the shape derivative in an extra step
% 		\item Filtering unphysical shape derivative contributions out
% 		\item Remeshing
% 		\item Mesh smoothing
% 		\item Using Riemannian metrics to induce geodesics on the manifold of admissible meshes
% 	\end{itemize}
% \end{frame}

\begin{frame}
	\frametitle{Results: Liquid Crystals}
	Given a starting shape \(\Omega_0\), we want to find a shape \(\Omega\) that minimizes the objective functional 
\begin{align}
    J(\Omega) = \int_{\Omega} |Q(x) - Q^*(x)|^2 \, \d x,
\end{align}
where \(Q^*\) is a target Q-tensor field.

Q-order tensor: describes how the liquid crystal molecules deviate from isotropy
	\begin{equation*}
		Q = S\left(\vec{n}(x) \otimes \vec{n}(x) - \frac{1}{3}\mathrm{Id}\right),
	\end{equation*}
	where \(S\) is the scalar order parameter and \(\vec{n}(x)\) the director field.
\end{frame}

\begin{frame}
	\frametitle{Results: Liquid Crystals}
	The Q-tensor field \(Q\) is constrained to be a minimizer of the Landau-de Gennes free energy functional
	\begin{align}
	E_{\textrm{LdG}}(Q) &= \int_{\Omega} \frac{l}{2} |\nabla Q|^2 \, \d x + \int_{\Omega} \left[ - \frac{a_B}{2} \operatorname{tr}(Q^2) - \frac{1}{3} \operatorname{tr}(Q^3) + \frac{1}{4} (\operatorname{tr}(Q^2))^2 \right] \, \d x \\
	&+ \frac{1}{2} \eta_{\partial \Omega} \int_{\partial \Omega} |Q - Q_{\partial \Omega}|^2 \, \mathrm{d}s.
	\end{align}
\end{frame}

\begin{frame}
	\frametitle{Results: Liquid Crystals}

	The target Q-tensor field \(Q^*\) is prescribed analytically:
\begin{align}
    Q^*(x) = S(r) \left( \vec{e}_r \otimes \vec{e}_r - \frac{1}{2}\mathrm{Id} \right) = S(r)\begin{pmatrix}
        \cos^2(\phi) - \frac{1}{2} &  \cos(\phi) \sin(\phi) \\
        \cos(\phi) \sin(\phi) & \sin^2(\phi) - \frac{1}{2}
\end{pmatrix}.
\end{align}
This is the minimizer for circle boundary conditions. We approximate \(S(r) = S_0\).

\end{frame}

\begin{frame}
	\frametitle{Results: Liquid Crystals}
	We start with an elliptical domain \(\Omega_0\):
	\begin{figure}
		\centering
		\includegraphics[width=0.8\textwidth]{LCEllipseInitial.png}
		\caption{Initial elliptical domain \(\Omega_0\).}
	\end{figure}
\end{frame}

\begin{frame}
	\frametitle{Results: Liquid Crystals}
\begin{figure}[h]
    \centering
	\includegraphics[width=0.95\textwidth]{LCEllipseIt08.png}
    \caption{Shape optimization result}
    \label{fig:LC-Ellipse-iterations}
\end{figure}
\end{frame}

\begin{frame}
	\frametitle{Results: Liquid Crystals}
	\begin{figure}
		\centering
		\begin{subfigure}{0.48\textwidth}
			\centering
			\includegraphics[width=\textwidth]{LCEllipseShapeDerivative.png}
			\caption{Unmodified Shape derivative }
			\label{fig:LC-shape-gradient-unmodified}
		\end{subfigure}
		\hfill
		\begin{subfigure}{0.48\textwidth}
			\centering
			\includegraphics[width=\textwidth]{LCEllipseShapeDerivativeProjNormal.png}
			\caption{Shape derivative projected onto normal}
			\label{fig:LC-shape-gradient-normal-proj}
		\end{subfigure}
		\caption{Visualization of the shape derivative for the liquid crystal shape optimization problem. Shown is the computed shape derivative \(j(x)\), with \(dJ(\Omega)[\theta] = \int_{\Omega} j(x) \cdot \theta(x) \, \d x\)}
	\end{figure}
\end{frame}

\begin{frame}
	\frametitle{Results: Liquid Crystals}
	Mesh quality can be improved by smoothing the shape derivative. Add a boundary smoothing term computing the shape gradient:
	\begin{align}
	a_\text{smooth}(u, v) = \delta_\Gamma \int_{\partial \Omega} \nabla_{\Gamma} u \cdot \nabla_{\Gamma} v \, \d s,
\end{align}
where \(\delta_\Gamma\) controls boundary smoothing and \(\nabla_{\Gamma}\) is the tangential gradient:
\begin{align}
	\nabla_{\Gamma} = \nabla - (\nabla \cdot n) n,
\end{align}
with \(n\) the unit normal.
\end{frame}


\begin{frame}
	\frametitle{Results: Liquid Crystals}
	\begin{figure}
		\centering
		\begin{subfigure}[t]{0.48\textwidth}
			\centering
			\includegraphics[width=\textwidth]{LCEllipseProjIt04.png}
			\caption{Final shape using normal projection}
			\label{fig:LC-shape-gradient-unmodified}
		\end{subfigure}
		\hfill
		\begin{subfigure}[t]{0.48\textwidth}
			\centering
			\includegraphics[width=\textwidth]{LCEllipseBeltramiIt06.png}
			\caption{Final shape using normal projection\\and additional smoothing}
			\label{fig:LC-shape-gradient-normal-proj}
		\end{subfigure}
		\caption{Comparison of optimized shapes after the final iteration for both approaches.}
	\end{figure}
\end{frame}


\begin{frame}
	\frametitle{Results: Liquid Crystals}
	As a second test case, we start with a rounded rectangle domain \(\Omega_0\):
	\begin{figure}
		\centering
		\includegraphics[width=0.8\textwidth]{LCRetctangleInitial.png}
		\caption*{Initial rounded rectangle domain \(\Omega_0\).}
	\end{figure}
\end{frame}

\begin{frame}
	\frametitle{Results: Liquid Crystals}
	\begin{figure}
		\centering
		\begin{subfigure}[t]{0.48\textwidth}
			\centering
			\includegraphics[width=\textwidth]{LCRectangleVerticalFinalDirector.png}
			\caption{Final shape with director field (vertical anchoring)}
			\label{fig:LC-shape-gradient-unmodified}
		\end{subfigure}
		\hfill
		\begin{subfigure}[t]{0.48\textwidth}
			\centering
			\includegraphics[width=\textwidth]{LCRectangleParallelItFinalDirector.png}
			\caption{Final shape with director field (parallel anchoring)}
			\label{fig:LC-shape-gradient-normal-proj}
		\end{subfigure}
		\caption{Comparison of optimized shapes after the final iteration for both anchoring conditions.}
	\end{figure}

\end{frame}

\begin{frame}
\frametitle{Conclusion and Outlook}
\begin{itemize}
    \item \textbf{Framework developed:} Flexible shape optimization using FEniCS and automated differentiation
    \item \textbf{Novel application:} Shape optimization for liquid crystals (Landau-de Gennes theory)
    \item \textbf{Key contributions:}
    \begin{itemize}
        \item Effective mesh quality maintenance techniques
        \item First step towards inverse design of LCE-based micro actuators
    \end{itemize}
    \item \textbf{Future directions:}
    \begin{itemize}
        \item Complex LC patterns (topological defects, 3D problems)
        \item Topology optimization (create/remove surfaces)
        \item More advanced mesh quality techniques
    \end{itemize}
\end{itemize}
\end{frame}

\begin{frame}
\frametitle{References}
    
\bibliographystyle{unsrt}
\bibliography{refs} 
\end{frame}

\begin{frame}[noframenumbering]
	\frametitle{Implementation}
	\begin{itemize}
		\item We implemented the shape optimization algorithm in Python using (legacy) FEniCS and the library dolfin-adjoint
		\item The state equations are solved using the finite element method (FEM) with a Lagrange P1 finite element space and a mesh of triangular elements representing the domain \(\Omega\)
		\item The shape derivative is calculated using the automated shape differentiation provided by dolfin-adjoint and the adjoint method, which involves solving an additional adjoint equation
	\end{itemize}
\end{frame}

\begin{frame}[noframenumbering]
	\frametitle{Nematic Liquid Crystals}
	Landau-de Gennes free energy
	\begin{align}
    E_{\textrm{LdG}}(\tensor{Q}) = \frac{1}{2}\int_{\Omega} W_{\textrm{el}}(\nabla \tensor{Q})\, \d x + \frac{1}{\eta_\mathrm{B} }\int_{\Omega} \psi_{\mathrm{bulk}}(\tensor{Q})\, \d x,
\end{align}
where the elastic energy density is given by
\begin{align}
    W_{\textrm{el}}(\nabla \tensor{Q}) = L|\Div \tensor{Q}|^2 + L_2 (\nabla \cdot \tensor{Q})^2 + L_3 \nabla \tensor{Q} : \nabla \tensor{Q}^T
\end{align}
and the bulk energy density
\begin{align}\label{eq:bulk-free-energy}
    \psi_{\mathrm{bulk}}(\tensor{Q}) = U_0 + \frac{A}{2} \tr(\tensor{Q}^2) + \frac{B}{3} \tr(\tensor{Q}^3) + \frac{C}{4} (\tr(\tensor{Q}^2))^2.
\end{align}
\end{frame}

\begin{frame}[noframenumbering]
	\frametitle{Results: Liquid Crystals}
	\begin{figure}
		\centering
		\begin{minipage}{0.48\textwidth}
			\centering
			\includegraphics[width=0.95\textwidth]{LCEllipseMeshQualIt01.png}
			\caption*{Iteration 1}
		\end{minipage}
		\hfill
		\begin{minipage}{0.48\textwidth}
			\centering
			\includegraphics[width=0.95\textwidth]{LCEllipseMeshQualIt02.png}
			\caption*{Iteration 2 (final)}
		\end{minipage}
		\caption{Shape optimization iterations using the elasticity inner product with added mesh quality penalty term}
		\label{fig:LC-Ellipse-MeshQuality-iterations}
	\end{figure}
\end{frame}

\begin{frame}[noframenumbering]
	\frametitle{Shape Gradients}
	\begin{itemize}
		\item Given a general Hilbert space \(H\) with inner product \(a_H(\cdot, \cdot)\), we can define the shape gradient \(g_\Omega\) of our functional \(J\) as the unique element in \(H\) such that
		\begin{align}
			a_H(g_\Omega, \nu) = - dJ(\Omega)[\nu] \quad \forall \nu \in H.
		\end{align}
		\pause
		\item \(\theta = g_\Omega \) is then a descent direction for the optimization, since for a sufficiently small step size \(\alpha\) we have
		\begin{align}
			J(\Omega + \alpha g_\Omega(\Omega)) &= J(\Omega) + \alpha dJ(\Omega)[g_\Omega] + o(\alpha)\\
			&= J(\Omega) - \alpha a_H(g_\Omega, g_\Omega) + o(\alpha) < J(\Omega).
		\end{align}
	\end{itemize}
\end{frame}

\begin{frame}[noframenumbering]
	\frametitle{Shape Gradients}
	We can choose different Hilbert spaces \(H\) to obtain different types of shape gradients:
	\begin{itemize}
		\item \(H = L^2(\Omega, \R^d)\) with \(a_H(u, v) = \int_\Omega uv \, \d x\)\pause
		\item \(H = H^1(\Omega, \R^d)\) with \(a_H(u, v) = \int_\Omega \nabla u : \nabla v + uv \, \d x\)\pause
		\item  \(H = H^1(\Omega, \R^d)\) with \(a_H(u, v) = \int_\Omega \sigma(u) : \epsilon(v) \, \d x\), where \(\epsilon(u) = \frac{1}{2}(\nabla u + (\nabla u)^T)\) is the linearized strain tensor and \(\sigma(u) = 2\mu \epsilon(u) + \lambda \trace(\epsilon(u))\mathrm{Id}\) is the stress tensor with Lamé parameters \(\mu, \lambda > 0\)
	\end{itemize}
\end{frame}
\end{document}


\begin{frame}[noframenumbering]
	\frametitle{Results: Liquid Crystals}
	\begin{figure}
		\centering
		\begin{minipage}{0.48\textwidth}
			\centering
			\includegraphics[width=\textwidth]{Presmesh_quality_comparison.png}
			\caption{Mesh quality comparison, lower values indicate better mesh quality}
		\end{minipage}
		\hfill
		\begin{minipage}{0.48\textwidth}
			\centering
			\includegraphics[width=\textwidth]{Presmain_objective_comparison.png}
			\caption{Objective value evolution}
		\end{minipage}
	\end{figure}
\end{frame}

\begin{frame}[noframenumbering]
	\frametitle{Results: Liquid Crystals}
	\begin{figure}
		\centering
		\includegraphics[width=0.6\textwidth]{Presvariance_radius_comparison.png}
		\caption*{Comparison of the variance of the boundary radius, lower values indicate shapes closer to a perfect circle.}
		\label{fig:LC-Ellipse-MeshQuality-Comparison}
	\end{figure}
\end{frame}


% Alignment needs an image, dont mention name


%  Wording Propose algorithm


%  Less text/words

% Add the target shape


%  LC put it in a flow charty thing
