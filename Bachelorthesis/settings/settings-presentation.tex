% ========================================================================================
% LaTeX Presentation Template by Lorenz Vogel (version 2022-03)
% Plehn Group, Institute for Theoretical Physics, Heidelberg University
% [compiler: pdfLaTeX]
% ========================================================================================

\usepackage[utf8]{inputenc} % input encodings (allow UTF-8 input)
\usepackage[T1]{fontenc} 	% selecting font encodings (use 8-bit T1 fonts)
\usepackage[english]{babel} % multilingual support (English language/hyphenation)
%\usepackage{lmodern} 		% Latin Modern fonts
\usepackage{geometry} 		% flexible and complete interface to document dimensions
%\usepackage{microtype} 		% subliminal refinements towards typographical perfection
\usepackage{amsmath} 		% math package (American Mathematical Society)
\usepackage{amssymb} 		% math package (extended symbol collection)

\newcommand\hmmax{0}
\newcommand\bmmax{0}

\usepackage{amsthm} 		% math package (typesetting theorems using AMS style)
\usepackage{mathtools} 		% math package (fixes various deficiencies of amsmath)
\usepackage{float} 			% floating objects such as figures and tables
\usepackage{graphicx} 		% enhanced support for graphics
\usepackage{tabularx} 		% tabulars with adjustable-width columns
\usepackage{booktabs} 		% professional-quality tables
\usepackage{pdfpages} 		% inclusion of external multi-page PDF documents
\usepackage{extarrows} 		% extra arrows beyond those provided in amsmath
\usepackage{multirow} 		% create tabular cells spanning multiple rows
\usepackage{multicol} 		% intermix single and multiple columns
\usepackage{caption} 		% customising captions in floating environments
\usepackage{subcaption} 	% support for sub-captions (captions for subfigures)
\usepackage{setspace} 		% set space between lines (single, onehalf, double)
\usepackage{xspace} 		% define commands that appear not to eat spaces
\usepackage{ragged2e} 		% alternative versions of ragged-type commands
\usepackage{stackrel} 		% enhancement to the \stackrel command
\usepackage{tikz} 			% create PostScript and PDF graphics
\usepackage{braket} 		% Dirac bra-ket notation
\usepackage{bm} 			% access bold symbols in maths mode
\usepackage{tensor} 		% typeset tensors (tensor-style super- and subscripts)
\usepackage{slashed} 		% slash through characters (Feynman slash notation)
\usepackage{siunitx} 		% SI units package (typesetting values with units)
%\usepackage{lastpage} 		% reference last page
\usepackage{cite} 			% improved citation handling
\usepackage[normalem]{ulem} % package for underlining
%\usepackage{fontawesome} 	% access to web-related icons
\usepackage{doi} 			% create correct hyperlinks for DOI numbers
\usepackage{hyperref} 		% hypertext links (handel cross-referencing commands)
\usepackage[dvipsnames]{xcolor} 				% colour management
\usepackage[most]{tcolorbox} 					% coloured and framed text boxes
%\usepackage[nameinlink, capitalize]{cleveref} 	% intelligent cross-referencing
%\usepackage[ruled, vlined]{algorithm2e} 		% floating algorithm environment
\usepackage{pgfpages}
\usepackage{isotope}
\usepackage{pifont}
\usepackage{appendixnumberbeamer}
\usepackage[version=4]{mhchem}
\usepackage[compat=1.0.0]{tikz-feynman}



%\usetheme{metropolis}
%\usecolortheme{dove}
%\usepackage[default, lining]{FiraSans}
%\usefonttheme[onlymath]{serif}

\usetheme{metropolis}
\usecolortheme{dove}
\usefonttheme[bitstream-charter]{serif}
\usepackage[bitstream-charter]{mathdesign}
\usefonttheme{professionalfonts}


\metroset{
	titleformat=regular, % regular, smallcaps, allsmallcaps, allcaps
	sectionpage=progressbar, % none, simple, progressbar
	subsectionpage=none, % none, simple, progressbar
	numbering=fraction, % none, counter, fraction
	progressbar=frametitle, % none, head, frametitle, foot
	%background=light, % dark, light
}




\newcommand{\cmark}{\ding{51}}
\newcommand{\xmark}{\ding{55}}

% prevent all line breaks in inline equations
\binoppenalty=10000
\relpenalty=10000

% fix \cal and \mathcal characters look (so it's not the same as \mathscr)
\DeclareSymbolFont{usualmathcal}{OMS}{cmsy}{m}{n}
\DeclareSymbolFontAlphabet{\mathcal}{usualmathcal}




\hypersetup{
	pdftitle={Bachelor Thesis Presentation},
	pdfauthor={David Li},
	colorlinks=true, 	% false: boxed links, true: colored links
	linkcolor={black}, 	% color of internal links (sections, pages, etc.)
	citecolor={black}, 	% color of citation links (links to bibliography)
	urlcolor={black}, 	% color of URL links (external links)
} 
% more hypersetup options: 
% linktoc=none,section,page,all (defines which part in the TOC is made into a hyperlink)
% hidelinks (removing color and border)





% algorithm environment settings
%\renewcommand{\algorithmautorefname}{Algorithm}
%\newcommand\mycommfont[1]{\textcolor{gray}{#1}}
%\SetArgSty{textnormal}
%\SetKwComment{Comment}{{\small\#}~}{}
%\SetCommentSty{mycommfont}

%\renewcommand{\thefootnote}{\fnsymbol{footnote}} 	% footnote symbol
%\setlength{\tabcolsep}{5pt}		% adding space between columns in a table
%\setlength{\parskip}{5pt} 		% parameter that characterises the paragraph spacing
%\setlength{\parindent}{0pt} 	% parameter that characterises the paragraph indentation

\usetikzlibrary{arrows, arrows.meta, shapes, calc}